\documentclass[a4paper,10pt]{article}
\usepackage[utf8x]{inputenc}
\usepackage{ucs}
\usepackage[french]{babel}
\usepackage[T1]{fontenc}
\usepackage{color}
\usepackage{geometry}
\usepackage{hyperref}
\usepackage[toc,page]{appendix}

\title{Un document en français}
\author{L'auteur}
\date{\today}

% Reduce default margins
\geometry{%
a4paper,
body={160mm,250mm},
left=25mm,top=25mm,
headheight=7mm,headsep=4mm,
marginparsep=4mm,marginparwidth=5mm}

% Set-up PDF properties
\makeatletter
  \hypersetup{
    pdftitle = {\@title},
    pdfauthor = {\@author}
  }
\makeatother

% Rename appendices
\renewcommand{\appendixname}{Annexes}
\renewcommand{\appendixpagename}{Annexes}
\renewcommand{\appendixtocname}{Annexes}

% Some useful commands
\newcommand{\todo}[1]{\fcolorbox{red}{yellow}{TODO: #1}}
\newcommand{\bksl}{\char`\\} % Backslash

\begin{document}
\maketitle

\section*{Résumé}

Ceci est le résumé du document.

\clearpage
\tableofcontents

\clearpage
\section*{Introduction}
\addcontentsline{toc}{section}{Introduction}

Bonjour, le monde!

\clearpage
\section{Première partie}
\subsection{Sous-partie I.1}
\subsubsection{Sous-sous-partie I.1.a}

\todo{Ceci est un marqueur "à faire"}

\subsubsection{Sous-sous-partie I.1.b}
\subsection{Sous-partie I.2}
\subsection{Sous-partie I.3}

\clearpage
\section{Seconde partie}
\subsection{Sous-partie II.1}

Pour utiliser un anti-slash comme \texttt{\bksl}, la commande est:
\begin{verbatim}
\char`\\
\end{verbatim}

\subsection{Sous-partie II.2}
\subsection{Sous-partie II.3}

\clearpage
\section{Troisième partie}
\subsection{Sous-partie III.1}
\subsection{Sous-partie III.2}
\subsection{Sous-partie III.3}

\clearpage
\section*{Conclusion}
\addcontentsline{toc}{section}{Conclusion}

Il faut bien conclure\footnote{même s'il reste les annexes...}

\clearpage
\begin{appendices}
\section{Première annexe}
\label{appendiceA}

\section{Seconde annexe}
\label{appendiceB}

\subsection{Sous-partie B.1}
\subsection{Sous-partie B.2}
\subsection{Sous-partie B.3}

\end{appendices}

\end{document}
